\chapter{Введение}

Задача агрегирования и классификации знаний для поиска литературы существует с момента появления больших библиотек. Используемые в библиотеках методы каталогизации и классификации книг \cite{Sill2007} существенно опираются на ручной труд и требуют привлечения экспертов для расширения или изменения структуры каталогов.

С другой стороны, обработка естественного языка позволяет автоматизировать решение многих задач работы с текстами. Так тематическое моделирование с момента своего создания успешно применялось для навигации по крупным текстовым корпусам и их визуализации \cite{Blei2007, Chaney2012, Chuang2012}. В больших текстовых коллекциях темы часто образуют иерархии, в которых каждая тема делится на более специфичные подтемы. Моделью такой ситуации являются иерархические тематические модели. Они удобны для навигации по коллекциям, поэтому являются подходящей моделью для агрегирования контента. 

В этой работе рассматривается задача построения иерархических тематических моделей на гетерогенных данных, собранных из различных источников. На сегодня нет общепринятых методов построения интерпретируемых тематических иерархий. Одна из проблем состоит в том, что для них не существует стандартных метрик качества, что делает невозможным сравнение различных моделей друг с другом. В работе предложены метрики качества тематических иерархий, согласованные с мнением людей об интерпретируемости иерархий. Для сравнения метрик с оценками людей проведен асессорский эксперимент. Кроме того, предложен метод построения иерархических тематических моделей гетерогенных данных для задачи агрегирования. Проведено сравнение нескольких алгоритмов и показано, что предложенный метод дает наилучшее качество и существенно превосходит базовый подход. 

Вероятностное тематическое моделирование --- это  раздел машинного обучения, решающий задачу поиска тем в коллекции документов. Тематическая модель определяет к каким темам относится каждый документ и какие слова образуют темы \cite{Blei2012}. Базовыми подходами построения тематических моделей являются PLSA \cite{PLSA} и LDA \cite{LDA}. В работах \cite{extLDA1,extLDA2, extLDA3} предложены модификаций данных подходов, учитывающие специфику конкретных задач. Аддитивная регуляризация тематических
моделей (ARTM) \cite{ARTM1, ARTM2, ARTM3, ARTM4} позволяет комбинировать различные модели, интерпретируя их как регуляризаторы в PLSA. 

Общепринятого определения и подхода к построению иерархических тематических моделей не существует. В модели иерархического LDA (hLDA) \cite{hLDA} темы образуют дерево. С другой стороны, модель иерархического распределения патинко (hPAM) \cite{hPAM} и модель иерархического ARTM (hARTM) \cite{hARTM} представляют собой направленный ациклический многодольный граф, что лучше соответсвует реальным отношениям между темами в мультидисциплинарных статьях.

В данной работе используется модель hARTM \cite{hARTM}. Это развитие идеи аддитивной регуляризации тематических моделей для задачи построения тематических иерархий.
Она позволяет применять регуляризацию как к темам всех уровней иерархии для комбинирования любых тематических моделей, так и к самой иерархии для контроля разреженности отношения <<родитель-ребенок>>.

В работах \cite{Mimno2011, Nikolenko2016, Nikolenko2017, Lau2014, Newman2010, Bouma2009} предложены различные метрики качества для отдельных тем в тематической модели. Большинство метрик используют различные меры близости наиболее вероятных слов темы для оценки ее качества. В таком случае качество всей модели -- это некоторая функция от качества ее тем, например, их среднее качество.

Тематическая иерархия состоит из тем и ребер иерархии, характеризующих связи между темами. Так как уже существуют принятые метрики качества тем, в данной работе для решения задачи оценки качества иерархии предлагается ввести метрики качества также для ребер иерархии. Предлагаемые метрики основаны на близости между наиболее вероятными словами темы-ребенка и темы-родителя. В асессорском эксперименте собраны мнения людей на тему наличия или отсутствия связи между темами. Показано, что предлагаемые метрики хорошо аппроксимируют человеческие оценки того, связаны темы некоторой пары или нет.

Базовый подход к построению общей тематической модели гетерогенных коллекций, собранной из нескольких источников, различных по объему и тематической структуре, предполагает слияние коллекций в одну и построение ее модели. При таком подходе темы, уникальные для меньшего из источников, теряются. Кроме того, для задачи агрегирования важно автоматически выбирать контент, который нужно включать в модель и отфильтровывать документы, не относящиеся к агрегируемому контенту. В данной работе для решения задачи агрегирования предлагается дополнять существующую качественную модель одного источника (базовой коллекции), построенную заранее, выборками документов из новых источников. Количество добавленных документов может во много раз превышать размер базовой коллекции. При этом документы, которые добавляются в модель, выбираются на основе близости к базовой коллекции. При обучении модели используется инициализации параметров модели дополненной коллекции параметрами модели базовой коллекции и сокращение словаря модели до словаря базовой коллекции. 
Такой подход позволяет сохранить качество модели и распределить новые документы по подходящим темам, сохранив при этом темы базовой коллекции. 


В вычислительных экспериментах в данной работе используются коллекции русскоязычного научно-популярного контента. Для проведения экспериментов используется BigARTM --- библиотека для тематического моделирования с открытым исходным кодом \cite{Vorontsov2015a, Frei2017}.