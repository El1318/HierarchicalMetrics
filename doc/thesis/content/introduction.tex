\chapter{Введение}
Вероятностное тематическое моделирование --- это  раздел машинного обучения, решающий задачу поиска тем в коллекции документов. Тематическая модель определяет, к каким темам относится каждый документ и какие слова образуют темы \cite{Blei2012}. Базовыми подходами построения тематических моделей являются PLSA \cite{PLSA} и LDA \cite{LDA}. В работах \cite{Chemudugunta2006, Rosen-Zvi2004, Than2012} предложены модификаций данных подходов, учитывающие специфику конкретных задач. Аддитивная регуляризация тематических
моделей (ARTM)\cite{ARTM1, ARTM2, ARTM3, ARTM4, Vorontsov2014} позволяет комбинировать упомянутые модели, интерпретируя их как регуляризаторы в PLSA. 


В больших текстовых коллекциях темы часто образуют иерархии, в которых каждая тема делится на более специфичные подтемы. Моделью такой ситуации являются тематические иерархии. Они удобны для навигации пользователей по коллекциям, поэтому являются подходящей моделью для агрегирования контента.

Общепринятого определения и подхода к построению иерархических тематических моделей не существует. В модели иерархического LDA (hLDA) \cite{hLDA} темы образуют дерево. С другой стороны, модель иерархического распределения патинко (hPAM)\cite{hPAM} и модель иерархического ARTM (hARTM) \cite{hARTM} представляют собой направленный ациклический многодольный граф, что лучше соответсвует реальным отношениям между темами в мультидисциплинарных научных и научно-популярных статьях.

Модель hARTM --- это развитие идеи аддитивной регуляризации тематических моделей для задачи построения тематических иерархий.
Она позволяет применять регуляризацию как к уровням иерархии для комбинирования любых тематических моделей, так и к самой иерархии для контроля разреженности отношения <<родитель-ребенок>>.

Цель данной работы --- построение иерархической тематической модели по нескольким источникам в рамках модели hARTM. Построение модели по объединенной коллекции источников, различных по объему и тематической структуре, не решает поставленной задачи, так как темы, уникальные для меньшего из источников, теряются. В работе предлагается дополнять существующую модель одного источника выборками документов из новых источников. Дополнение происходит в два этапа: сначала выбираются документы нового источника, наиболее подходящие для добавления в коллекцию, затем выбранные документы добавляются в коллекцию и строится дополненная модель. При этом в силу стратегии инциализации и регуляризации модели существующие темы сохраняются, а добавленные документы уточняют соответсвующие им темы на первом уровне иерархии. На втором уровне добавленные документы могут порождать подтемы, характерные только для нового источника. 

В работах \cite{hetHTM1, ametyst} рассмотрено построение иерархических тематических моделей гетерогенных источников . Однако, 
задача последовательного достроения модели затронута в работах, не использующих подход вероятностного тематического моделирования \cite{IncHClustering}.
Предложенный в данной работе подход выигрывает у построения по объединенной коллекции по качеству решения и скорости.

Для проведения экспериментов используется BigARTM --- библиотека для тематического моделирования с открытым исходным кодом \cite{Vorontsov2015a, Frei2017}.