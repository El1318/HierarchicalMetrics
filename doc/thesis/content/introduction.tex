\chapter{Introduction}
Energy efficiency turns to be the one of main challenge for humanity. Lack of resources together with rocketed energy demand across the planet urge peoples to increase efficiency of their energy systems. One of the most promising technologies in this field is Smart Grid. Traditionally, the term "grid" denotes an electricity system, that supports electricity generation, transmission, distribution and control. Most of them use for direct energy delivery from several large generators to consumers. In contrast, a Smart Grid is an electricity grid which utilise two-way flows of energy and information to establish automated and distributed next-generation energy delivery network \cite{Fang2012}. These intelligent technologies are incorporated across the entire system which improve its efficiency, safety and reliability \cite{Gao2012}. 

The main advantage of Smart Grid systems is a potential for altering end-user consumption behaviour, which allows to shift peak demands to stabilise daily consumption profile. Such interaction between energy operators and consumers is called Demand-Response (henceforth DR). Plenty of DR schemes have been proposed recent years, and some of them are in use in US energy market. They differ in what they assume about the system, what type of grid architecture they utilise, what type of motivation for end users to participate into the program they propose, and what mathematical approach to emerge system's intelligence they apply. The main difficulty which most of the schemes are trying to tackle is a lack of communication between server and customers in real world, which prohibits to address each device individually. However, most of works assume two-sided information stream, which leads to significant infrastructure investment therefore detaining proliferation of such intelligent systems into the market. 

In this study we propose a novel scheme which is aimed to minimise communication requirements without a lessening in functionality to curtail or adjust overall consumption in grid. In particular, it require to know only aggregated energy consumption in each period, which is significantly more realistic then two-sided communication architecture. To pursue this we consider the set of consumers as an ensemble of devices, which behaviour is representable by finite Markov chains. This is an extension of \cite{Chertkov2017} ideas, who proves this approach to be viable end efficient. To tackle optimisation subroutines we consider this problem as a Linear Contextual Bandit With Knapsack (BwK \cite{Badanidiyuru2013}) setup, applying the algorithms proposed in \cite{Agrawal2015}. The main contribution of this work is proposed markov-chains-based ensemble model, which provides context for contextual bandit learning algorithms and has a naturally linear dependency of reward over this context. It enables to apply advanced bandit algorithms as \cite{Agrawal2015} without loss of its convergence guarantees making no additional assumptions. 

\section{Demand-Response Overview}


One of the distinguishing features of Smart Grid networks is demand manageability. This concept of the Demand-Side Management (DSM) includes all activities aiming to alter the consumer's demand profile to make it match the supply or to effectively incorporate renewable energy sources \cite{Alizadeh2012}. Nowadays the major activity in DSM is Demand-Response (DR)\cite{Palensky2011}. According to the United States Department of energy, Demand-Response is "a tariff or program established to motivate changes in electric use by end-use customers, in response to changes in the price of electricity over time, or to give incentive payments designed to induce lower electricity use at times of high market prices or when grid reliability is jeopardized" \cite{DepartmentofEnergyUSA2006}. As was summarised in \cite{Vardakas2015}, the main objectives of the application of DR are:
\begin{itemize}
    \item Reduction of the total energy consumption both on demand and transmission sides. Such overall consumption curtailment may help governments and energy providers to meet their pollution obligations \cite{DepartmentofEnergyUSA2006, Shishlov2016, UnitedNations/FrameworkConventiononClimateChange2015}.
    \item Reduction of the maximal needed power generation in order to eliminate the need of activating expensive-to-run power plants to meet peak demands. 
    \item Efficient incorporation of renewable energy sources through making the demand follow the available local supply fluctuations. Such incorporation may significantly increase the overall system's reliability in regions with high penetration of wind farms and solar panels \cite{Santacana2010}. 
    \item Reduction or even elimination of overloads in distribution systems shifting peak demands' time for a subset of consumers.
\end{itemize} 

Amount of DR service providers (also know as Aggregators) in US energy market soared up recent years due to emerging new intelligent solution and overall market liberalisation \unsure{is it true?} \todo{Add citation}. A typical aggregator company represents several thousand households or a few dozens of commercial consumers (e.g. downtown office buildings). \missing{some statistics} One of the most typical aggregator's business models is providing   

As shown in figure \missing{insert scheme} the principal DR-scheme consists of cooperation of four main participants: a) an Aggregator, b) a System Operator (SO), c) Power Generation Unit(s) d) and Power Consumer(s) \cite{Medina2010}. Their interaction is a cyclic process typically started by the SO, which determines the preferred power consumption and sends it to the Aggregator. Next the Aggregator chooses participating loads from available, calculates possible change in demand and sends it back to the SO. And finally the Operator informs the most available substations about the upcoming demand. In such scheme the Aggregator provides the grid's intelligence executing optimisation procedures pr revealing problems in distribution system \cite{Vardakas2015}. 

DR Schemes distinct in control architecture they utilise, and in motivation to participate which they provide to customers.

\paragraph{DR Schemes by its architecture} DR schemes may be classified into centralised and distributed programs \cite{Zhou2012}, according to where the decision for the execution program are made. In centralised schemes load activations are managed only by the central utility. Such schemes are easy to implement, but turn to be hard-headed in large and complex systems. However it remains an effective approach for controlling ensembles of termostatically controlled loads \cite{Hao2015}, charging systems for electro vehicles \cite{Yano2012} and commercial consumers \cite{Motegi2007}. For example \missing{Add example}


\paragraph{DR Schemes by its motivation}
The proposed motivation schemes usually adopt either price-based or incentive-based approach. 





\section{Multiarmed Bandits}
Reinforcement learning can be defined as a learning paradigm concerned with learning to control a complex system so to maximize a numerical performance measure that express some long-term objective \cite{Szepesvari2010}. The most typical setting where reinforcement learning operates is an iterative process of agent-environment interactions (see fig \textit{somefig}). Formally, at the moment $t$ the agent performs an action $a_t$ according to his current policy $\pi_t$ and gets a reward $r_t$ from the environment. The aim of the agent is to maximize the reward $\sum_{t = t_0}^{t_0 + T}r_t$ in a given time horizon $T$, or, equally, to minimize the regret $\bar{R_T} = R^x*_T - \sum_{t = t_0}^{t_0 + T}r_t$ where $R^*_T$ is the best reward the agent can theoretically get with the best policy $\pi^*$ available. However, in most of the real problem we have some additional information associated with each arm or with the situation in general. The vector that represents this information is usually called <<the context>> hence the related setup is called <<contextual multi armed bandit>>. Under the described conditions the agent should take into account the context to achieve better performance. 
 
Multiarmed bandit problem can be defined as a Reinforcement Learning setup with a fixed state of the environment, so the problem here is to learn the best policy. 

The amount of papers devoted to Reinforcement Learning has soared up recent years due to significant advances in \textit{some domains (link)}.

Plenty of papers is devoted to the general multi armed bandit setup. One of the most famous algorithms -- $\varepsilon$-greedy -- was proposed by \cite{Auer2002}: it pulls an arbitrary arm with a probability of $\varepsilon$ and the best arm according to the current policy otherwise. This is due tho the exploration-exploitation balance problem: without this arbitrary steps the agent may fail to gain enough statistics for determining the best policy. However, $\varepsilon$ is a hyperparameter which should be appropriately tuned. Another approach here is so-called <<optimism in the face of uncertainty>> \cite{Lai1985} according to which the learner should choose an arm with the best Upper Confidence Bound. A very successful algorithm which has implemented this technique is UCB1 \cite{Auer2002}, later analysed and improved by \cite{Audibert2009}. The later not only often outperforms UCB1, but also had shown to be essentially unimprovable under the assumption that the variance of the reward associated with some of the actions are small.

In contrast to the regular bandit problem, the contextual bandit setup tends to be tougher problem to solve. One approach is to assume the particular dependency model between the arms' context $x_{a,t}$ and the expected reward $r_t$. One possible model is a linear one: \begin{align}
        &\label{eq:reward_linear_assumption}
        r_{a,t} = \bar{r_{a,t}} + \varepsilon_{a,t} \\
        s.t.\, & \bar{r_{a,t}}  = \E[r_{t,a}|x_{t,a}] = x_{t,a}^T\theta^* \\
        & \E \varepsilon_{a,t}  = 0 
    \end{align}
    
    For this linear case \cite{Li2010} proposed an adaptation of UCB1 approach: LinUCB. Later \cite{Abbasi-Yadkori2011} provided better theoretical analysis by eliminating the assumption that the reward is identically distributed over time and arms, which is mostly far from true. The similar case with different regularization strategy was examined by \cite{Auer2003}. Another heuristics for balancing exploration and exploitation is known as Thompson Sapling was adapted to the linear reward model by \cite{Agrawal2013}. 

\section{Reinforcement Learning in Demand-Response}