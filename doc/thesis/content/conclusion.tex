\chapter{Заключение}

В этой работе было предложено несколько автоматических метрик качества для отношений «родитель-ребенок» в иерархической тематической модели. Было показано, что метрика качества $\mathrm{S}_{emb}$, основанная на векторных представлениях слов, хорошо аппроксимирует мнения асессоров о том, существует ли связь между темами. Другие метрики продемонстрировали меньшую, но приемлемую согласованность с асессорами.

Кроме того, были предложены два подхода к измерению качества иерархии в целом, основанные на усреднении качества ребер и на измерении качества ранжирования ребер, которое задает модель.

Для решения задачи агрегирования было предложено несколько модификаций базового метода построения моделей гетерогенных текстовых коллекций: фильтрация, сокращение словаря, инициализация. Было проведено сравнение качества моделей, построенных с использованием этих модификаций. Показано, что предлагаемый алгоритм, использующий все три модификации, дает наилучшее качество модели.

Результаты работы были представлены на 24-ой международной конференции
по компьютерной лингвистике и интеллектуальным технологиям ("Диалог" 2018) и на 60-й Научной конференции МФТИ.
